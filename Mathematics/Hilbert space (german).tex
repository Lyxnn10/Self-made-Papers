\documentclass{scrreprt}
\usepackage[utf8]{inputenc}
\usepackage{amsmath}
\usepackage{amssymb}
\usepackage{amsfonts}
\usepackage{amstext}
\usepackage{amsthm}
\usepackage{mathtools}
\usepackage{braket}
\usepackage{tikz}
\usetikzlibrary{mindmap,trees}
\usepackage{verbatim}
\usepackage{xcolor}
\usepackage{multirow}
\usepackage{iitem}
\usepackage[T1]{fontenc} 
\usepackage[ngerman]{babel}
\usepackage{hyphenat}
\hyphenation{Mathe-matik wieder-gewinnen}
\usepackage{tcolorbox}
\usepackage[Glenn]{fncychap}
\usepackage{setspace}
\onehalfspacing
\usepackage{hyperref}
\hypersetup{
    colorlinks=true,
    linkcolor=black,
    filecolor=magenta,      
    urlcolor=cyan,
    pdftitle={Overleaf Example},
    pdfpagemode=FullScreen,
    }
\urlstyle{same}
\usepackage[top=1in,bottom=1in,left=1.5in,right=1.5in]{geometry}
\usepackage{graphicx, soul, tikz}


\title{\huge $\textsc{Der Hilbertraum}$}
\author{\large $\textsl{Lyxnn}$}
\date{\large $\textsl{31.10.2021}$}

\begin{document}

\maketitle


\chapter{Hilbertraum}

Es sei $\mathbb{K} \in \{\mathbb{R}, \mathbb{C}\}$.


\section{Defnition und einführende Beispiele}
$\textbf{Definition 1.1}$  (inneres Produkt, Skalarprodukt) $\newline \newline$
Ein $\underline{inneres}$ $\underline{Produkt}$ auf einem $\mathbb{K}$-Vektorraum V ist eine positiv definite hermitesche Sesquilinearform $\langle\cdot, \cdot\rangle: V \times V \rightarrow \mathbb{K}$, das heißt: $\forall x, y, z \in V$ und $\forall\alpha \in \mathbb{K}$ gilt: $\newline$

\begin{itemize}
    \item[1)] Positiv Definit: $\langle x, x\rangle \geq 0$ und $\langle x, x\rangle = 0 \Leftrightarrow x=0$.
    \item[2)] Hermitesch: $\langle x, y\rangle = \overline{\langle y, x\rangle}.$
    \item[3)] Sesquilinear: $\langle x+\alpha y,  z\rangle = \langle x, z \rangle + \overline{\alpha}\langle y, z\rangle$ und $\langle x,y+\alpha z\rangle = \langle x, y\rangle + \alpha \langle x, z\rangle$.
\end{itemize}

$\newline \newline$
$\textbf{Bemerkung 1.2}$
$\newline \newline$
Das Skalarprodukt kann auch linear im ersten und semilinear im zweiten Argument definiert werden. Im Hinblick auf die Bra-Ket-Notation wird es im Folgenden wie oben definiert verwendet, da damit das Skalarprodukt als $\braket{x|y}$ geschrieben werden kann, wobei $\bra{x}$ Bra und $\ket{y}$ Ket genannt wird. $\newline$
Die Abbildung $\bra{x}: V\rightarrow \mathbb{K},$ $y\mapsto \braket{x|y}$ kann als Linearform (lineare Abbildung vom Vektorraum $V$ in den zugrundeliegenden Körper $\mathbb{K}$) auf $V$ aufgefasst werden.
$\newline \newline$
$\textbf{Definition 1.3}$ (Prähilbertraum, Innenproduktraum) $\newline \newline$
Sei $V$ ein $\mathbb{K}$-Vektorraum und sei $\langle\cdot, \cdot\rangle: V \times V \rightarrow \mathbb{K}$ ein inneres Produkt, so ist $(V,\langle\cdot, \cdot\rangle)$ ein $\underline{Prähilbertraum}$.
$\newline \newline$
$\textbf{Definition 1.4}$ (Hilbertraum)
$\newline \newline$
Ein $\underline{Hilbertraum}$ ist ein bezüglich der vom inneren Produkt induzierten Norm $||\cdot|| = \sqrt{\langle\cdot, \cdot\rangle}$ vollständiger Prähilbertraum.
$\newline \newline$
$\textbf{Beispiel 1.5}$
$\newline \newline$

\begin{itemize}
    \item[1.]  komplexer n-dimensionaler Koordinatenraum mit Standardskalarprodukt: $(\mathbb{C}^{n},\braket{x,y} = \sum_ {k=1} ^{n} \overline{x}_{k}y_{k})$.
    \item[2.] Folgenraum $\ell^{2}(\mathbb{N})=\left\{\left(x_{n}\right){n} \subset \mathbb{C} ; \sum_{k=1}^{\infty}\left|x_{k}\right|^{2}<\infty\right\}^{2}$ mit $\newline$ $\braket{x,y} = \sum_ {k=1} ^{n} \overline{x}_{k}y_{k}$.
    \item[3.] Lebesgue-Raum $L_2$. $\newline$
    Sei $(S,\mathcal{A},\mu)$ ein Maßraum, $\mathcal{L}_{2}(S,\mathcal{A},\mu):=\left\{f: S \rightarrow \mathbb{K}, \int_{S}|f|^{2} \mathrm{~d} \mu<\infty\right\}$, und $\mathcal{N}=\left\{f \in \mathcal{L}_{2} \mid f=0 \mu-\right.$ fast überall $\}$, so ist $L_{2}$ der Quotientenraum $\mathcal{L}_{2}/\mathcal{N}$. $\newline$
    Mit $\newline$
    \begin{center}
        $\begin{aligned}\huge \braket{f,g}_{L_{2}}= \int_ {S} \braket{f(x),g(x)}d\mu (x)\end{aligned}$
    \end{center}
    $\newline$ ist ein inneres Produkt definiert, wobei das Skalarprodukt im Integral, das Standardskalarprodukt bezeichne; $\newline$ die Vollständigkeit liefert der aus der Maßtheorie Satz von Riesz-Fischer.
\end{itemize}

\section{Orthonormalbasen}
$\textbf{Definition 1.6}$ (Orthonormalsystem)
$\newline \newline$
Eine Teilmenge $\mathcal{E}$ eines Hilbertraums $\mathcal{H}$ heißt $\underline{Orthonormalsystem}$, falls $||e|| = 1$ $\forall e \in \mathcal{E}$ und $\braket{e,f} = 0$ $\forall e, f \in \mathcal{E}$ mit $e\neq f$.
$\newline \newline$
$\textbf{Definition 1.7}$ (Orthonormalbasis)
$\newline \newline$
Ein Orthonormalsystem $\mathcal{E}$ heißt $\underline{Orthonormalbasis}$ (oder vollständiges Orthonormalsystem) von $\mathcal{H}$, falls $span(\mathcal{E})$ ($span(\mathcal{E}):=\left\{\sum_{i=1}^{n} \lambda_{i} a_{i} \mid \lambda_{i} \in \mathbb{K}, e_{i} \in \mathcal{E}, n \in \mathbb{N}\right\}$) dicht in $\mathcal{H}$ liegt (also $\overline{span(\mathcal{E})} = \mathcal{H}$).
$\newline \newline$
$\textbf{Bemerkung 1.8}$
$\newline \newline$
Damit lässt sich jedes $x \in \mathcal{H}$ als Grenzwert einer Folge in $span(\mathcal{E})$ Schreiben.
$\newline \newline$
$\textbf{Definition 1.9}$ (orthogonales Komplement)
$\newline \newline$
Sei $V$ eine Teilmenge eines Hilbertraums $\mathcal{H}$.$\newline$ $V^{\perp}:=\{w \in \mathcal{H} \mid \forall v \in V:\langle v, w\rangle=0\}$ heißt
das orthogonale Komplement von $V$. 
$\newline \newline$
$\textbf{Satz 1.10}$ (Charakterisierung einer Orthonormalbasis)
$\newline \newline$
Sei $\mathcal{E}=\left\{e_{k} \mid k \in \mathbb{N}\right\}$ ein abzählbares Orthonormalsystem in einem Hilbertraum $\mathcal{H}$. $\newline$ Dann sind die folgenden vier Aussagen äquivalent:

\begin{itemize}
    \item[1.] $\mathcal{E}^{\perp} = \{0\}$.
    \item[2.] $\mathcal{E}$ ist Orthonormalbasis.
    \item[3.] Es gilt $\forall x\in \mathcal{H}:$ 
    \begin{center}
        $\begin{aligned}x=\sum_ {k=1} ^{\infty} \braket{e_{k},x}e_{k}\end{aligned}.$
    \end{center}
    \item[4.] Es gilt $\forall x, y \in \mathcal{H}$
    $\newline$
    \begin{center}
        $\begin{aligned}\braket{x,y}= \sum_ {k=1} ^{\infty}\braket{x,e_{k}}\braket{e_{k},y}\end{aligned}.$
    \end{center}
    \end{itemize}
    $\newline$
    
$\textbf{Beweis.}$ $"(1) \Rightarrow(2)": U:=\overline{span(\mathcal{E})}.$ $\mathcal{E} \subset U$ liefert $U^{\perp} \subset \mathcal{E}^{\perp}$, also $U^{\perp}=\{0\}$. $\newline$
Sei nun $x \in \mathcal{H}$. Definiere $s_{n}=\begin{aligned}
\sum_{k=1}^{n} \left\langle e_{k}, x\right\rangle e_{k}\end{aligned}$. Dann ist $\left(s_{n}\right)_{n}$ eine Cauchyfolge, da gilt:

\begin{center}
$
\begin{aligned}
\left\|s_{n}-s_{m}\right\|^{2} &=\left\|\sum_{k=m+1}^{n}\left\langle e_{k}, x\right\rangle e_{k}\right\|^{2}=\left\langle\sum_{j=m+1}^{n}\left\langle e_{j}, x\right\rangle e_{j}, \sum_{k=m+1}^{n}\left\langle e_{k}, x\right\rangle e_{k}\right\rangle=\\
&=\sum_{k=m+1}^{n}\left\langle e_{k}, x\right\rangle\left\langle\sum_{j=m+1}^{n}\left\langle e_{j}, x\right\rangle e_{j}, e_{k}\right\rangle=\sum_{k=m+1}^{n}\left\langle e_{k}, x\right\rangle \sum_{j=m+1}^{n} \overline{\left\langle e_{j}, x\right\rangle} \delta_{j k}=\\
&=\sum_{k=m+1}^{n}\left|\left\langle e_{k}, x\right\rangle\right|^{2} \leq \sum_{k=m+1}^{\infty}\left|\left\langle e_{k}, x\right\rangle\right|^{2} \rightarrow 0 \text { für } m \rightarrow \infty.
\end{aligned}
$
\end{center}
Damit existiert $\begin{aligned}
s=\lim_{n \rightarrow \infty} s_{n} \in U\end{aligned}$. Nun folgt aber mit der Stetigkeit des Skalarprodukts

\begin{center}
    $\begin{aligned}\forall k:\left\langle s-x, e_{k}\right\rangle=\lim_{n \rightarrow \infty}\left\langle s_{n}-x, e_{k}\right\rangle=0.\end{aligned}$
\end{center}
$\newline$


Insbesondere ist $s-x \in U^{\perp}$, also $s=x$ und damit $x \in U$. Nun folgt $U=\mathcal{H}$. $"(2) \Rightarrow(3)":$ Die Partialsumme $s_{n}=\sum_{k=1}^{n}\left\langle e_{k}, x\right\rangle e_{k}$ fiir $x \in \mathcal{H}, n \in \mathbb{N} .$ $\forall 1 \leq j \leq n:\left\langle e_{j}, s_{n}-x\right\rangle=\left\langle e_{j}, s_{n}\right\rangle-\left\langle e_{j}, x\right\rangle=\sum_{k=1}^{n}\left\langle e_{k}, x\right\rangle\left\langle e_{j}, e_{k}\right\rangle-\left\langle e_{j}, x\right\rangle=0$
Damit folgt: $\forall y \in span\left(e_{1}, \ldots, e_{n}\right): y \perp s_{n}-x$, also mit dem Satz des Pythagoras (anwendbar, da $x-s_{n} \perp s_{n} -y$ und letzteres aus $span(e_{1},\cdots,e_{n}))$ : $\newline$ $\left.\|x-y\|^{2}=\left\|x-s_{n}\right\|^{2}+\left\|s_{n}-y\right\|^{2} \geq\left\|x-s_{n}\right\|^{2}.$ $\newline$
Abschließend ist eine Folge $(x_{n})_{n} \rightarrow x$ und eine monoton steigende Folge $(m_{n})_{n} \subset \mathbb{N}$, so gewählt, dass $x_{n} \in span(e_{1}, \ldots, e_{m_{n}})$. $\newline$ Dann ist 0 \leq\|x-s_{m_{n}}\|^{2} \leq\|x-x_{n}\|^{2} \rightarrow 0.$ Wegen $\|x-s_{n+1}\|^{2} \leq\|x-s_{n}\|^{2}$ konvergiert $(s_{n})_{n}$ gegen $x$, was die Behauptung liefert. $\newline$ $"(3)\Rightarrow(4) ":$ wieder $s_{n}=\sum_{k=1}^{n}\langle e_{k}, x\rangle e_{k}$ und $t_{n}=\sum_{k=1}^{n}\langle e_{k}, y\rangle e_{k}$. Dann folgt: $\langle s_{n}, t_{n}\rangle=\sum_{k=1}^{n}\langle x, e_{k}\rangle\langle e_{k}, y\rangle$. Wegen $s_{n} \rightarrow x$ und $t_{n} \rightarrow y$ folgt mit der Stetigkeit des Skalarprodukts die Behauptung. $\newline$
$"(4) \Rightarrow (1)":$ Nun $y=x$. Es folgt: $\|x\|^{2}=\sum_{k=1}^{\infty}|\langle x, e_{k}\rangle|^{2}$. Sei $x \in \mathcal{E}^{\perp}$. Dann gilt $\forall k \in \mathbb{N}:$ $\langle x, e_{k}\rangle=0$ und damit folgt $\|x\|^{2}=0$ also $x=0.$

$\newline \newline$
$\textbf{Definition 1.11}$
$\newline \newline$
Existiert für einen Hilbertraum $\mathcal{H}$ eine abzählbare Orthonormalbasis $\mathcal{E}$, dann nennt man den Hilbertraum
$\mathcal{H}$ separabel.






\end{document}
