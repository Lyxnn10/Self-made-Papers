\documentclass{scrreprt}
\usepackage[utf8]{inputenc}
\usepackage{amsmath}
\usepackage{amssymb}
\usepackage{amsfonts}
\usepackage{amstext}
\usepackage{amsthm}
\usepackage{mathtools}
\usepackage{braket}
\usepackage{tikz}
\usetikzlibrary{mindmap,trees}
\usepackage{verbatim}
\usepackage{xcolor}
\usepackage{multirow}
\usepackage{iitem}
\usepackage[T1]{fontenc} 
\usepackage[ngerman]{babel}
\usepackage{hyphenat}
\hyphenation{Mathe-matik wieder-gewinnen}
\usepackage{setspace}
\onehalfspacing
\usepackage[Glenn]{fncychap}

\title{\huge$\textbf{Black-Scholes-Differentialgleichung}$}
\author{\large $\textsl{Lyxnn}$}
\date{\large $\textsl{27.11.2021}$}

\begin{document}

\maketitle

\chapter{Black-Scholes-Differentialgleichung}

\section{Herleitung der Black-Scholes Differentialgleichung}

Bewertungsformel für europäische Optionen:
\begin{itemize}
    \item Der risikolose Zinssatz $r$ ist bekannt und bleibt auch in Zukunft konstant.
    \item Dividendenzahlungen der Aktie sind ausgeschlossen.
    \item Die Option sei europäisch. 
    \item Beim als ständig angenommenen Handel mit Wertpapieren fallen keine Transaktionskosten bzw. Steuern an.
    \item Leerverkäufe seien nicht eingeschränkt. Die daraus resultierende Mittel dürfen für den zukünftigen Handel von Wertpapieren verwendet werden.
    \item Alle verfügbaren Wertpapiere seien beliebig teilbar.
    \item Der Aktienkurs $S$ entspricht einem stochastischen Prozess in stetiger Zeit, dessen Varianzrate proportional zu $S^2$ sei. $S$ sei auf jedem Intervall log-normalverteilt, wobei die Varianz des Ertrages als konstant angenommen wird.
\end{itemize}

Aus dem letzten Punkt ergibt sich unmittelbar die folgende Differentialgleichung für den Aktienkurs $S$: $\newline$
\begin{equation}
    \begin{aligned} dS(t) = \mu S(t)dt+\sigma S(t) dW(t)   \end{aligned},
\end{equation} 
$\newline$
wobei die Konstante $\mu$ für die jährlich erwartete prozentuelle Rendite der Aktie steht und die Volatilität $\sigma$ für deren Standardabweichung. $\newline$ Es ist klar, dass einem Ito-Prozess in der differenzialschreibweise entspricht. $\newline$ Es muss insofern auf der linken Seite vor den Differentialen jeweils $S(T)$ multiplikativ berücksichtigt werden, als $\mu$ nur einem relativen Anteil entspricht. $\newline \newline$ Bei höheren Aktienkursen ist klarerweise auch eine höhere absolute Änderung des Aktienkurses bei konstantem $\mu$ zu erwarten.
$\newline \newline$
Die Lösung der stochastischen Differentialgleichung ist eine geometrische Brownsche Bewegung:
$\newline$
\begin{equation}
    S(t)=S(0)\exp^{(\mu -\frac{1}{2} \sigma^2)t+\sigma W(t)}
\end{equation}
$\newline$
Da der Preis der europäischen Option durch eine Funktion des Aktienkurses und der Zeit modelliert werden soll, können wir das Lemma von Ito auf $f(S(t),t)$ anwenden, wobei S(t) der Ito-Prozess ist. $\newline$
Wir erhalten, wobei der Übersichtlichkeit halber hier und im Folgenden auf das Anschreiben der Argumente von $S$ und $f$ verzichtet wird: 

$\newline$
\begin{equation}
    \begin{aligned} df=\left(\frac{\partial f}{\partial S} \mu S+\frac{\partial f}{\partial t}+\frac{1}{2}\frac{\partial^2 f}{\partial S^2} \sigma^2 S^2\right)dt+\frac{\partial f}{\partial S}\sigma SdW(t) 
    \end{aligned}.
\end{equation}
$\newline$

Das Ziel ist jetzt, den risikobehafteten Teil in der oberen Gleichung zu eliminieren. Dazu bilden wir ein Hedging-Portfolio, wobei das Derivat als Short-Position gehalten werden soll. In jedem Portfolio soll also eine Option der Quantität -1 und

\begin{equation}
    \begin{aligned}\frac{\partial f}{\partial S}   \end{aligned}
\end{equation}
Anteile der Aktie, also long, gehalten werden.
 $\newline$
Es handelt sich hierbei um Delta-Hedging. Für den Wert des Portfolios $P$ gilt dann
$\newline$
\begin{equation}
    \begin{aligned}P=-f+\frac{\partial f}{\partial S}S  \end{aligned}.
\end{equation}
$\newline$
Für die Änderung des Portfoliowertes in einem infinitesimalen Zeitraum folgt also damit:
$\newline$
\begin{equation}
    \begin{aligned}dP=-df+\frac{\partial f}{\partial S}dS \end{aligned}.
\end{equation}
$\newline$
Analoge Überlegungen könnten auch für kleine Zeitintervalle $\Delta t$ angestellt werden. $\newline$ In die letzte Gleichung können wir nun (0.1) und (0.3) einsetzen und nach anschließendem Vereinfachen erhaltet man: 

$\newline$
\begin{equation}
    \begin{aligned}dP=- \left(\frac{\partial f}{\partial t} + \frac{1}{2}\frac{\partial^2 f}{\partial S^2}\sigma^2 S^2\right) dt.  \end{aligned}
\end{equation}
$\newline$
Hier wurden sogar die Parameter $\mu$ eliminiert worden, weshalb man daraus schließen kann, dass die Änderung des Portfoliowertes in einem infinitesimalen Zeitraum unabhängig von der erwarteten Aktienrendite ist. $\newline$ Da auch das Differential $dW(t)$ eliminiert wurde, kann man daraus folgern, dass das Portfolio $P$ für den infinitesimalen Zeitraum risikolos sein muss. Folglich muss für $dP$ auch gelten:
$\newline$
\begin{equation}
    \begin{aligned}dP=rPdt, \end{aligned}
\end{equation}
$\newline$
wobei $r$ dem risikolosen Zinssatz entspricht. $P$ muss hier also auch dieselbe Rendite wie jedes andere risikolose Wertpapier liefern. $\newline$
Ansonsten würde dies der Annahme der Arbitragefreiheit widersprechen, da abhängig von der Höhe der Rendite des Portfolios im Vergleich zu $r$ ein Kaufen bzw. Shorten des Portfolios eine Möglichkeit für risikolosen Gewinn bieten würde. $\newline$
Verwendet man dies gemeinsam mit (0.7), so ergibt sich nach dem Einsetzen von $P$ sowie elementaren Umformungen mit anschließendem Weglassen des Differentials $dt$, das multiplikativ für alle Terme beider Seiten gilt, die Black-Scholes-Merton Differentialgleichung:
$\newline$
\begin{equation}
    \begin{aligned} rS\frac{\partial f}{\partial S}+\frac{\partial f}{\partial t}+\frac{1}{2}\sigma^2 S^2 \frac{\partial^2 f}{\partial S^2}=rf\end{aligned}.
\end{equation}






\end{document}
